\documentclass{muratcan_cv}

\setname{Cody}{Bloemhard}
\setaddress{Utrecht/The Netherlands}
\setmobile{codyb.xyz}
\setmail{cblmhrd@live.nl}
\setposition{Student} %ignored for now
\setlinkedinaccount{https://linkedin.com/in/codybloemhard} %you can play with color of the template (red is also nice..)
\setgithubaccount{https://github.com/ocdy1001} %you can play with color of the template (red is also nice..)
\setthemecolor{red} %you can play with color of the template (red is also nice..)

\begin{document}
% Set variables
% You can add sections, texts, explanations just by copying the style below. Replace the dummy texts "\lipsum[1][x-x]\par" with actual texts.
% Create header
\headerview
\vspace{1ex}
% Sections
%
% Summary
\addblocktext{Summary}{%
I am a computer science student and former Art student. My focus is on software engineering/development.
I have worked on all kinds of things including 2D and 3D games, game engines, rendering engines, GPGPU, command line applications, open source libraries, android apps, data processing tools, experimental prototypes, interactive electronic devices, AI's, procedural generated content and algorithmic art.
I care about combining art and programming and I support the philosophy, use and development of free and open source software.
}
%
% Education
\section{Education}
    \datedexperience{MSc in Computing Science}{2021-2023}
    \explanation{Utrecht University}
    \explanationdetail{\coloredbullet\ %
     Ongoing
    }
    \datedexperience{BSc in Computer Science}{2017-2021}
    \explanation{Utrecht University}
    \datedexperience{Propedeuse Bachelor of Creative Media and Game Technologies, Game Development}{2015-2017}
    \explanation{HKU (University of the Arts Utrecht)}
    \explanationdetail{\coloredbullet\ %
     During this time I worked with many Designers and Artists in contrast with University where there are only CS students, showing me a whole new world.
    }
%
% Experience
\section{Experience}
    \datedexperience{Utrecht Companion to the Earth}{2020-2020}
    \explanation{Software Engineer}
    \explanationdetail{\coloredbullet\ %
     Utrecht Companion to the earth is an app that has the goal to aid Geo-science students. Our client for the project was the Geo-science department of Utrecht University. My responsibility was optimizing large amounts of data for mobile use and building the Android app. Technologies used by me were Rust and Kotlin with Android Studio. We used Agile methodologies and test automation with CI. I learned about building Android apps, Big Data, compression and the Geo-science field.
     \href{https://github.com/ocdy1001/uu-uce}{src}.
    }
%
% Skills
\section{Skills}
    %
    \newcommand{\skillone}{\createskill{Programming Languages}{\textbf{\emph{Experienced:}} \ \  Rust \cpshalf C\# \cpshalf \textbf{\emph{Familiar:}} \ \  C/C++ \cpshalf Java/Kotlin \cpshalf Python \cpshalf Haskell \cpshalf GLSL \cpshalf HTML/CSS/JS}}
    %
    \newcommand{\skilltwo}{\createskill{Frameworks \ \& \ Libraries}{ OpenGL \cpshalf OpenCl \cpshalf \textbf{\emph{Previous:}} \ \ Unity3D \cpshalf Monogame \cpshalf Gamemaker \cpshalf Arduino}}
    %
    \newcommand{\skillthree}{\createskill{Miscellaneous}{Git \cpshalf Vim \cpshalf Agile \cpshalf Arch/Artix/Void Linux}}
    %
    \newcommand{\skillfour}{\createskill{Languages}{\textbf{\emph{Native:}} \ \  Dutch \ \ \textbf{\emph{Professional:}} \ \ English }}
    %
    \createskills{\skillone, \skilltwo, \skillthree, \skillfour }
%
% Experience
\section{Some of my Projects}
    \newcommand{\pra}{ GPGPU Raytracer in Rust, C\# and OpenCl: \href{https://github.com/ocdy1001/clrays}{src}. }
    \newcommand{\prb}{ CPU Raytracer in C\# with models, stratisfied sampled area lights, textures, stochastic glossy reflections, refraction, HDR skyboxes, SSAA, FXAA, multithreading: \href{https://github.com/ocdy1001/UU_Raytracer}{src}. }
    \newcommand{\prc}{ Pacman clone... But in Haskell: \href{https://github.com/ocdy1001/fp-pacman}{src}. }
    \newcommand{\prd}{ Linux rice, custom desktop environment by configuring and forking many open source sub-components: \href{https://github.com/ocdy1001/LinuxRice}{src}. }
    \newcommand{\pre}{ Procedural terrain generation and hydraulic erosion: \href{https://github.com/ocdy1001/worldgen2}{src}. }
    \newcommand{\prg}{ Term-basics-linux: A library for extra terminal IO on linux: \href{https://github.com/ocdy1001/term-basics-linux}{src}. }
    \newcommand{\prh}{ MIDI Music Generator: \href{https://github.com/ocdy1001/GenerateMidiMusic}{src}.}
    \newcommand{\pri}{ HackerRank challenges (Problem Solving(Basic), C++(Basic) certified): \href{https://www.hackerrank.com/ocdy1001}{link}. }
    \newcommand{\prj}{ Personal Planner: A TUI app to help plan your life: \href{https://github.com/ocdy1001/PersonalPlanner}{src}.}
    \newcommand{\prk}{ \href{www.codyb.xyz}{codyb.xyz}: my personal website: \href{https://github.com/ocdy1001/ocdy1001.github.io}{src}. }
    \newcommand{\prl}{ Shapebar: a statusbar for x11. Forked from lemonboy's bar, merged and expanded to look slick while being minimal: \href{https://github.com/ocdy1001/shapebar}{src}. }
    \newcommand{\prm}{ Miscellaneous work and visual art: \href{https://www.instagram.com/cbloemhard/}{instagram} }
    \newcommand{\prn}{ Frag: library in Rust to render fragment shaders to screen or video. \href{https://github.com/ocdy1001/frag}{src} }
    \newcommand{\pro}{ TheøryFrøg: A Rust/WASM webapp to query music theory info. \href{https://github.com/ocdy1001/theory-frog}{web src},
    \href{https://github.com/ocdy1001/music-theory}{lib src}, \href{https://codyb.xyz/theory-frog.html}{web app}}
    \newcommand{\listofextras}{\pro, \prn, \pra, \prb, \prc, \prd, \pre, \prg, \prh, \pri, \prj, \prk, \prl }
    %
    \createbullets{\listofextras}
%
%Footnote
\createfootnote
\end{document}
