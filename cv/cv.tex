\documentclass{muratcan_cv}

\setname{Cody}{Bloemhard}
\setaddress{Utrecht/The Netherlands}
\setmobile{codyb.xyz}
\setmail{cblmhrd@live.nl}
\setposition{Student} %ignored for now
\setlinkedinaccount{https://linkedin.com/in/codybloemhard} %you can play with color of the template (red is also nice..)
\setgithubaccount{https://github.com/ocdy1001} %you can play with color of the template (red is also nice..)
\setthemecolor{red} %you can play with color of the template (red is also nice..)

\begin{document}
%Set variables
%You can add sections, texts, explanations just by copying the style below. Replace the dummy texts "\lipsum[1][x-x]\par" with actual texts.
%Create header
\headerview
\vspace{1ex}
%Sections
%
% Summary
\addblocktext{Summary}{%
I am a computer science student. I have worked on all kinds of things including 2D and 3D games, game engines, rendering engines, GPGPU, command line applications, open source libraries, android apps, data processing tools, experimental prototypes, interactive electronic devices, AI's and procedural generated content.
}
%
%Education
\section{Education}
    \datedexperience{Utrecht University}{2017-2020}
    \explanation{(not yet) B.S in Computer Science}
    \explanationdetail{\coloredbullet\ %
     Finishing bachelor next quarter.
    }
    \datedexperience{HKU (Hogeschool voor de Kunsten Utrecht)}{2015-2017}
    \explanation{Propedeuse Bachlor of Creative Media and Game Technologies, Game Development}
    \explanationdetail{\coloredbullet\ %
     During this time I worked with many Designers and Artists in contrast with University where there are only CS students.
    }
%
% Experience
\section{Experience}
    \datedexperience{Utrecht Companion to the Earth}{2020-2020}
    \explanation{Software Engineer}
    \explanationdetail{\coloredbullet\ %
     Utrecht Companion to the earth is an app that has the goal to aid Geo-science students. Our client for the project was the Geo-science department of Utrecht University. My responsibility was optimizing large amounts of data for mobile use and building the Android app. Technologies used by me were Rust and Kotlin with Android Studio. I learned about building Android apps, Shapefile and KML files, compression and the Geo-science field.
    }
%
% Skills
\section{Skills}
    %
    \newcommand{\skillone}{\createskill{Programming Languages}{\textbf{\emph{Experienced:}} \ \  Rust \cpshalf C\# \cpshalf \textbf{\emph{Familiar:}} \ \  C/C++ \cpshalf Java/Kotlin \cpshalf Python \cpshalf Haskell \cpshalf GLSL \cpshalf HTML/CSS/JS}}
    %
    \newcommand{\skilltwo}{\createskill{Miscellaneous}{Git \cpshalf Vim \cpshalf Agile \cpshalf Arch/Artix/Void Linux}}
    %
    \newcommand{\skillthree}{\createskill{Frameworks \ \& \ Libraries}{ OpenGL \cpshalf OpenCl \cpshalf \textbf{\emph{Previous:}} \ \ Unity3D \cpshalf Monogame \cpshalf Gamemaker \cpshalf Arduino}}
    %
    \newcommand{\skillfour}{\createskill{Languages}{\textbf{\emph{Native:}} \ \  Dutch \ \ \textbf{\emph{Professional:}} \ \ English }}
    %
    \createskills{\skillone, \skilltwo, \skillthree, \skillfour}
%
% Experience
\section{Some of my Projects}
    \newcommand{\prone}{ GPGPU Raytracer in Rust,C\# and OpenCl. }
    \newcommand{\prtwo}{ CPU Raytracer in C\# with models, stratisfied sampled area lights, textures, stochastic glossy reflections, refraction, HDR skyboxes, SSAA, FXAA, multithreading}
    \newcommand{\prthree}{ Pacman clone... But in Haskell }
    \newcommand{\prfour}{ Linux rice, custom desktop environment by configuring and forking many open source sub-components }
    \newcommand{\prfive}{ Procedural terrain generation and hydraulic erosion }
    \newcommand{\prsix}{ Shapefile-linter: a tool to optimize geological data for mobile use }
    \newcommand{\prseven}{ A library for extra terminal IO on linux }
    \newcommand{\preight}{ MIDI Music Generator }
    \newcommand{\listofextras}{\prone, \prtwo, \prthree, \prfour, \prfive, \prsix, \prseven, \preight}
    %
    \createbullets{\listofextras}
%
%Footnote
\createfootnote
\end{document}
